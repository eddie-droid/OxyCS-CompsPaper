\documentclass[10pt,twocolumn]{article} 

\usepackage{oxycomps} % use the main oxycomps style file

\bibliography{references}

\pdfinfo{
    /Title (Literature Review)
    /Author (Eddie Valdez)
}

\title{Oxy CS Comps Paper}

\author{Eddie Valdez}
\affiliation{Occidental College}
\email{evaldez@oxy.edu}

\begin{document}

\maketitle

\begin{abstract}
    This document serves as the start of my Oxy CS Comps paper. My Oxy Comps project will consist of an application that I develop that will help users integrate mindfulness into their daily lives. I will include the Problem Context, the Technical Background, and the Prior Work of my Oxy CS Comps project. In the future when I do more work towards my Comps project I will be adding a Methods, Evaluations, and Ethical Considerations section.
\end{abstract}

\section{Problem Context}


The main reason why I want to develop this application is because I believe practicing mindfulness is essential to living less stressful lives by changing how we react and understand our emotions. There continues to be ever growing research that shows the numerous benefits that mindfulness has on our health and overall well-being \citetitle{Keng2011Effects}. This is important because now more than ever people are experiencing higher rates of stress, burnout, anxiety, and depression especially because of the COVID-19 Pandemic. Most people would like to experience a life without all these mental phenomenon but it seems impossible not to. While this is true, many people do not realize that there is an alternative approach which is to be able to control how we react and engage with these negative mental phenomenon. 

Many people try to be mindful, however usually people typically only do it for a short period of time and then forget to keep practicing. This irregular practice of mindfulness makes it hard to be truly mindful in our daily lives. This is why I believe a mindfulness application that intentionally attempts to be regularly integrated into people's daily lives could be really effective in building the habit of mindfulness. I personally have experienced the negative effects of stress, anxiety, and depression and I found that a  daily integration of mindfulness in my life keeps these mental phenomenons from negatively affecting my life. When I first began my mindfulness journey, I found it really hard to ever practice being mindful. I would simply get distracted by everyday tasks and rarely ever remembered to engage in mindfulness practices like mediation, gratitude, or nature walks. It was not until I started using Meditation apps that I began seeing the benefits of mindfulness. However, most mediation apps are limited to solely meditation and while that was a good start, I believe I was not fully engaged with mindfulness practices until I also started to practice mindfulness more holistically on my own. This includes practicing gratitude, spending time in nature, regular meditative reflections throughout the day, and general mindfulness reminders. I want these aspects of mindfulness that really helped me and have been shown to be beneficial to others to be key features in the application that I develop. These are the features that I want my application to have because I believe they can help more people become more mindful by making it easier to engage in Mindfulness practices holistically. 

The biggest challenges I think I will have to tackle are making an application that is well organized and designed so that I can be as simple as possible for a user to engage in mindfulness practices and get them to want to keep using the application. I think the design and content is very important because I want the application to be appealing to as many people as possible which means it should have a variety of features to practice mindfulness and they should be well organized. It should be an enjoyable experience overall which is why it should be easy and appealing to navigate the different mindfulness practice tools.
\section{Technical Background}



For my application, I will likely develop the application using Java or Swift. I want to use one of these two because most people either have an Android or an iOS phone so I want the application to be able to be used by as many people as possible. In my application I will create a simple API that is focused on a user friendly UI. I will not have to take user information. The app will just have content that is all locally stored and content that can requested from a back-end server. The app will be likely organized into interfaces, abstract classes, and classes of the different sections and their content. The main content that will probably be stored on a server will be the guided meditations.
In the application, I want to have three different sections. one of them will be a map that contains personalized locations in nature that are good to practice mindfulness in, a section that allows you to practice meditating, and an ideas section that contains general tips and information about how to be mindful. I do not know what the code will be like to navigate the different sections and their content but I imagine conditions will be checked like if a section is touched and then it will open that section object will all the content. The app will be object oriented and I will find ways to make it a very aesthetic app.  

\section{Prior Work}



Recently with the rise in awareness of the importance of mental health, many applications have been created to promote mindfulness. Some of the most popular apps that are related to what I want to develop are Headspace, Calm, Breethe, Reflectly, and Waking Up. Most of these apps however, are essentially just mediation timers, guided meditations, and reminders to meditate. Apps that consist of just these features have shown very little evidence for their efficacy in developing mindfulness. They all mainly revolve around meditation and that is where I want my app to stand out. I want my application to include all aspects of mindfulness like a section for you to spend time in nature through a map with different locations, an information section where you can read about what mindfulness even is and how to practice it, reminders throughout the day regarding mindfulness tips like to slow down or focusing on one thing at a time along with wise inspirational quotes, and also a simple relaxing timer for you to be able to meditate easily along with prompts to reflect on. 

Every application mentioned above is very difficult to navigate because there is so much content. With that means the apps aren't very enjoyable to navigate which negatively effects the user experience with the app. I want my app to only have a 3 different sections with not an overwhelming amount of content. Because of this, I think my app will be a lot more visually appealing. A key difference in why there will be a lot less content is because I will not have guided meditations be the focus of my app. This is the main reason why Headspace, Calm, Breethe, and Waking Up are not easy to navigate: because they have an overwhelming variety of Guided Meditations that are all showcased for the user to select the best one for them. I believe this is ineffective as it never made me want to use them. There is also evidence that this makes these apps less effective \cite{Eysenbach2015Review}. An application that has a simpler user interface is Reflectly. This application is mostly an AI journal but its design is aesthetically pleasing and it doesn't include guided meditations. I also like that this app has daily positive quotes along with sending short helpful messages throughout the day so I want to implement similar features in my application.

Many apps have different aspects of what I want to create but none of them encompass all my desired features. Importantly, non of them implement a feature that includes locations where one can go to spend time in nature and be mindful there. I think this is an essential feature because it can be difficult to find these kinds of locations on your own and spending time in nature is an important part of developing mindfulness. I believe that these changes can help a user of my application to develop mindfulness because they are the things that helped me develop mindfulness for myself and there is evidence that some of these features I am wanting to implement can be effective in developing mindfulness.

\section{Methods}
I will start developing my app this summer. I do not have plan of the exact steps of doing this. I know I want to finish by early September or October.

\section{Evaluation}
I will be evaluating my application using a randomized controlled trail. Mindfulness training, which involves observing thoughts and feelings without judgment or reaction, has been shown to improve aspects of psychosocial well-being when delivered via in-person training programs such as mindfulness-based stress reduction (MBSR) and mindfulness-based cognitive therapy (MBCT). However, less is known about the efficacy of digital training mediums, such as smartphone apps. In order to test my app using a controlled trail, I will assign novice meditators to be randomly allocated to my mindfulness  application or to a psychoeducational audiobook control featuring an introduction to the concepts of mindfulness and meditation. The interventions will be delivered via the same mindfulness app where they will be matched across a range of criteria. They will also be presented to participants as well-being programs.At the end, evaluations of improvements in mental health aspects will be measured. In order to do this, I will need to develop two different applications. This will also have to take at least 3 weeks for users to use the app. This means I should be done with the app in early September or October.

Another method of evaluation that I can conduct is using the MARS quality rating paired with evaluations of psychosocial well-being improvement over time using several criteria like irritability, affect, stress resulting from external pressure, and reducing stress associated with personal vulnerability. The MARS quality rating comprises the four main subscales: (1) user engagement (entertainment, interest, customization, interactivity, target group), (2) functionality ( performance, usability, navigation, gestural design), (3) aesthetics ( layout, graphics, visual appeal), and (4) information quality (accuracy of app description, goals, quality of information, quantity of information, quality of visual information, credibility, evidence base).

If I do this I should also have a completed project by early September or October because it will take time to conduct these trails. 

\printbibliography 

\end{document}
